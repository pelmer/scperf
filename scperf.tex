\documentclass [draft,notitlepage] {article}
\usepackage[margin=1.0in,nohead,nofoot]{geometry}
\usepackage{url}
\usepackage{hyperref}
\pagestyle{empty}
\newcommand\myurl[2]{\url{#1}}
\newcommand\email[1]{\href{mailto:#1}{\nolinkurl{#1}}}

\title{CMS Software and Computing Performance Paper (Title TBD)}

\author{CMS Collaboration - Editors: K.Bloom and P.Elmer}


\begin{document}

\maketitle

\section{Introduction}

\subsection {Physics goals} 

\begin{itemize}
\item describe quantitatively the scale required to reach physics goals
\item Higgs~\cite{CMSHIGGS} as possible example physics problem, should (briefly!) run through all the S\&C needs to do the physics analysis.  (At the very least, Higgs provides the example of quick turnaround of physics -- when was the last data taken before 4/7/12 Higgs announcement?)  Is there another physics problem that has some contrasting requirements?  Another physics requirement -- must get results out fast due to competition and great community interest!
\end{itemize}

\subsection{The Large Hadron Collider and the CMS detector}

\begin{itemize}
\item Scale and requirements of the LHC
\item Scale and requirements of the CMS detector
\item Event size, Pile-Up (e.g. effect on multiplicities, etc.)
\end{itemize}

\subsection{Software and Computing System} 

\begin{itemize}
\item Describe the major limitations and challenges in building the S\&C system
\item requirements of distributed development and distributed computing facilities:
\item These include: large group of code developers whose are ultimately novice coders, wide geographical distribution of developers, highly distributed computing resources too, newfangled grid system that had to be shaken down, very distributed and ultimately novice-coder analysts! (I think we should be covering analysis in here too)...what else do we need to add to this list?
\end{itemize}

Goals of software and computing:

\begin{itemize}
\item Software development model must incorporate the work of many geographically distributed coders
\item Software must run on all necessary architectures, must be robust against potentially fragile computing facilities
\item Software needs to be able to run at a scale needed to turn around results quickly
\item Computing must run the software at the sufficient scale
\item Computing must make the best use of all available resources
\item Computing must make the data and computing resources available to all analysts
\item In general, software and computing should never limit the rate of the production of physics results and papers (``factory'' latency) Do we have a way of demonstrating this actually happened?
\end{itemize}


\section{Software Applications} 
Describe what are we trying to do with the computing:
\begin{itemize}

\item Trigger (how much of this is ``us''?  HLT?)
% CHEP 2013 	"The CMS High Level Trigger" D. Troncino
% https://cms-mgt-conferences.web.cern.ch/cms-mgt-conferences/conferences/pres_display.aspx?cid=1085&pid=7050

% CHEP12 "The CMS High Level Trigger System: Experience and Future
% Development" A. Sparatu
% https://cms-mgt-conferences.web.cern.ch/cms-mgt-conferences/conferences/pres_display.aspx?cid=665&pid=4520

\item Reconstruction
% CHEP13 "The Role of Effective Event Reconstruction in the Higgs Boson Discovery at CMS", S. Krutelyov
% https://cms-mgt-conferences.web.cern.ch/cms-mgt-conferences/conferences/pres_display.aspx?cid=1085&pid=7258

\item Analysis

\item MC simulation
% CHEP13 - there is both a fastsim talk and a fullsim talk, but I think
% we combined these. I'm confused by what we have in Cinco, will double
% check on CHEP13 site (I'm one of the track coordinators for that CHEP track)

\item Calibration/alignment
% CHEP13 "Alignment and calibration of CMS detector during collisions at LHC" R.Castello
% https://cms-mgt-conferences.web.cern.ch/cms-mgt-conferences/conferences/pres_display.aspx?cid=1085&pid=7170

\item I considered also things like data management here, but that strictly speaking isn't something needed to get a physics result, which I think is what this section is about.  Anything missing from the list?
\end{itemize}


\section{Software Implementation}
\begin{itemize}

\item Software development model

\item Framework architecture 
% Perhaps some of the text can come from the paper about the new
% framework evolution:
% CHEP13 "Stitched Together: Transitioning CMS to a Hierarchical Threaded Framework" C. Jones
% https://cms-mgt-conferences.web.cern.ch/cms-mgt-conferences/conferences/pres_display.aspx?cid=1085&pid=7110
% or perhaps there are older CHEP papers.

\item Software architecture and evolution

\item ``Performance'' numbers (code base size, number of developers, etc.) 
over time, major releases

\item Software and Release validation
% CHEP13 "The Rise of the Build Infrastructure" G.Eulisse
% https://cms-mgt-conferences.web.cern.ch/cms-mgt-conferences/conferences/conf_display.aspx?cid=1085

\item Evolution with architectures (32bit/64bit, compilers)
% Various ACAT/CHEP presentations from P.Elmer, M.Kortelainen

\item CPU and I/O optimization?
% Various ACAT/CHEP presentations from P.Elmer, M.Kortelainen

\item Documentation?
% Surely we have documentation?
% CHEP12 "Developing CMS software documentation system" M. Stankevicius
% https://cms-mgt-conferences.web.cern.ch/cms-mgt-conferences/conferences/pres_display.aspx?cid=665&pid=4506

\end{itemize}


\section{Computing Implementation} 
(Not sure if ordering is quite right..)


\begin{itemize}

\item Distributed Model (details in practice including tier system, numbers)

\item Prompt Calibration Loop (PCL)
% CHEP12 "Handling of time-critical Conditions Data in the CMS experiment -
% Experience of the first year of data taking" G. Govi
% https://cms-mgt-conferences.web.cern.ch/cms-mgt-conferences/conferences/pres_display.aspx?cid=665&pid=4571

\item Operations Model (does this include site operations?)
% CHEP13 "CMS Computing Operations During Run1" O.Gutsche
% https://cms-mgt-conferences.web.cern.ch/cms-mgt-conferences/conferences/pres_display.aspx?cid=1085&pid=7055

% CHEP12 "Towards a global monitoring system for CMS computing operations"
% A. Sciaba
% https://cms-mgt-conferences.web.cern.ch/cms-mgt-conferences/conferences/pres_display.aspx?cid=665&pid=4445
% CHEP12 "CMS Tier-0: Preparing for the future" D. Hufnagel
% https://cms-mgt-conferences.web.cern.ch/cms-mgt-conferences/conferences/pres_display.aspx?cid=665&pid=4424

\item Data transfers/network usage
% CHEP13 "The CMS Data Management System" N. Magini
% https://cms-mgt-conferences.web.cern.ch/cms-mgt-conferences/conferences/pres_display.aspx?cid=1085&pid=7057

% CHEP12 "CMS Data Transfer operations after the first years of LHC
% collisions" R. Kaselis
% https://cms-mgt-conferences.web.cern.ch/cms-mgt-conferences/conferences/pres_display.aspx?cid=665&pid=4537

\item Data quality monitoring (or is this a software topic?)

\item Databases
% CHEP12  "Comparison of the Frontier Distributed Database Caching System with NoSQL Databases" D. Dykstra
% https://cms-mgt-conferences.web.cern.ch/cms-mgt-conferences/conferences/pres_display.aspx?cid=665&pid=4418

% CHEP12 "CMS experience with online and offline Databases" A. Pfeiffer
% https://cms-mgt-conferences.web.cern.ch/cms-mgt-conferences/conferences/pres_display.aspx?cid=665&pid=4444

\item Data Management (design choices, implementation and performance)

\item Workflow Management (design choices, implementation and performance)
% CHEP12 "The CMS workload management system" S. Wakefield
% https://cms-mgt-conferences.web.cern.ch/cms-mgt-conferences/conferences/pres_display.aspx?cid=665&pid=4454

% CHEP12 "A new era for central processing and production in CMS"
% E. Fajardo Hernandez
% https://cms-mgt-conferences.web.cern.ch/cms-mgt-conferences/conferences/pres_display.aspx?cid=665&pid=4402

\item User Analysis (+ Support??)
% CHEP12 "CMS Analysis Deconstructed" S. Malik
% https://cms-mgt-conferences.web.cern.ch/cms-mgt-conferences/conferences/pres_display.aspx?cid=665&pid=4399

\item ``Performance'' numbers (resource usage, data sizes, analysis participation, turnaround times if that can be captured) [Or should this be integrated with other sections, somehow?  Editorial choice to consider.]
\end{itemize}

\section{Anticipated Evolution for Run 2}
\begin{itemize}
\item Multithreaded framework
%CHEP12 "Study of a Fine Grained Threaded Framework Design" C. Jones
% https://cms-mgt-conferences.web.cern.ch/cms-mgt-conferences/conferences/pres_display.aspx?cid=665&pid=4396

%CHEP12 "Multi-core processing and scheduling performance in CMS"
%J. Hernandez
% https://cms-mgt-conferences.web.cern.ch/cms-mgt-conferences/conferences/pres_display.aspx?cid=665&pid=4365

\item Software engineering efforts
%CHEP12 "Development and Evaluation of Vectorised and Multi-Core Event
%Reconstruction Algorithms within the CMS Software Framework" D. Piparo
% https://cms-mgt-conferences.web.cern.ch/cms-mgt-conferences/conferences/pres_display.aspx?cid=665&pid=4546

\item Evolution of tiered computing model
% CHEP12 "Evolution of the Distributed Computing Model of the CMS
% experiment at the LHC" C. Grandi
% https://cms-mgt-conferences.web.cern.ch/cms-mgt-conferences/conferences/pres_display.aspx?cid=665&pid=4347

\item Use of data federations, changes in data distribution
%CHEP12 "Implementing data placement strategies for the CMS experiment
%based on a popularity mode" D. Giordano
% https://cms-mgt-conferences.web.cern.ch/cms-mgt-conferences/conferences/pres_display.aspx?cid=665&pid=4443
\item Efforts on opportunistic resources
% CHEP12 "Controlled overflowing of data-intensive jobs from oversubscribed
% sites" I. Sfiligoi
% https://cms-mgt-conferences.web.cern.ch/cms-mgt-conferences/conferences/pres_display.aspx?cid=665&pid=4419
\end{itemize}

\section{Conclusion}
Would be good to get back to the physics -- discuss cases where we got results out quick (Higgs), turned around new samples quickly, got new releases or calibrations out fast.  These are ultimately the measures of our success!  Or put another way, here is where we should clearly emphasize that we’ve met some of the goals described earlier.




\newpage
\bibliographystyle{unsrt}
\bibliography{references}

\end{document}

