\section{Introduction}

\subsection {Physics goals} 

\begin{itemize}
\item describe quantitatively the scale required to reach physics goals
\item Higgs~\cite{CMSHIGGS} as possible example physics problem, should (briefly!) run through all the S\&C needs to do the physics analysis.  (At the very least, Higgs provides the example of quick turnaround of physics -- when was the last data taken before 4/7/12 Higgs announcement?)  Is there another physics problem that has some contrasting requirements?  Another physics requirement -- must get results out fast due to competition and great community interest!
\end{itemize}

\subsection{The Large Hadron Collider and the CMS detector}

\begin{itemize}
\item Scale and requirements of the LHC
\item Scale and requirements of the CMS detector
\item Event size, Pile-Up (e.g. effect on multiplicities, etc.)
\end{itemize}

\subsection{Software and Computing System} 

\begin{itemize}
\item Describe the major limitations and challenges in building the S\&C system
\item requirements of distributed development and distributed computing facilities:
\item These include: large group of code developers whose are ultimately novice coders, wide geographical distribution of developers, highly distributed computing resources too, newfangled grid system that had to be shaken down, very distributed and ultimately novice-coder analysts! (I think we should be covering analysis in here too)...what else do we need to add to this list?
\end{itemize}

Goals of software and computing:

\begin{itemize}
\item Software development model must incorporate the work of many geographically distributed coders
\item Software must run on all necessary architectures, must be robust against potentially fragile computing facilities
\item Software needs to be able to run at a scale needed to turn around results quickly
\item Computing must run the software at the sufficient scale
\item Computing must make the best use of all available resources
\item Computing must make the data and computing resources available to all analysts
\item In general, software and computing should never limit the rate of the production of physics results and papers (``factory'' latency) Do we have a way of demonstrating this actually happened?
\end{itemize}


\section{Software Applications} 
Describe what are we trying to do with the computing:
\begin{itemize}

\item Trigger (how much of this is ``us''?  HLT?)
% CHEP 2013 	"The CMS High Level Trigger" D. Troncino
% https://cms-mgt-conferences.web.cern.ch/cms-mgt-conferences/conferences/pres_display.aspx?cid=1085&pid=7050

% CHEP12 "The CMS High Level Trigger System: Experience and Future
% Development" A. Sparatu
% https://cms-mgt-conferences.web.cern.ch/cms-mgt-conferences/conferences/pres_display.aspx?cid=665&pid=4520

\item Reconstruction
% CHEP13 "The Role of Effective Event Reconstruction in the Higgs Boson Discovery at CMS", S. Krutelyov
% https://cms-mgt-conferences.web.cern.ch/cms-mgt-conferences/conferences/pres_display.aspx?cid=1085&pid=7258

\item Analysis

\item MC simulation
% CHEP13 - there is both a fastsim talk and a fullsim talk, but I think
% we combined these. I'm confused by what we have in Cinco, will double
% check on CHEP13 site (I'm one of the track coordinators for that CHEP track)

\item Calibration/alignment
% CHEP13 "Alignment and calibration of CMS detector during collisions at LHC" R.Castello
% https://cms-mgt-conferences.web.cern.ch/cms-mgt-conferences/conferences/pres_display.aspx?cid=1085&pid=7170

\item I considered also things like data management here, but that strictly speaking isn't something needed to get a physics result, which I think is what this section is about.  Anything missing from the list?
\end{itemize}


\section{Software Implementation}
\begin{itemize}

\item Software development model

\item Framework architecture 
% Perhaps some of the text can come from the paper about the new
% framework evolution:
% CHEP13 "Stitched Together: Transitioning CMS to a Hierarchical Threaded Framework" C. Jones
% https://cms-mgt-conferences.web.cern.ch/cms-mgt-conferences/conferences/pres_display.aspx?cid=1085&pid=7110
% or perhaps there are older CHEP papers.

\item Software architecture and evolution

\item ``Performance'' numbers (code base size, number of developers, etc.) 
over time, major releases

% CHEP10 "The CMS Reconstruction Software" D. Lange
% https://cms-mgt-conferences.web.cern.ch/cms-mgt-conferences/conferences/pres_display.aspx?cid=462&pid=2346

\item Software and Release validation
% CHEP13 "The Rise of the Build Infrastructure" G.Eulisse
% https://cms-mgt-conferences.web.cern.ch/cms-mgt-conferences/conferences/conf_display.aspx?cid=1085

% CHEP 10 "Release Strategies: CMS approach for Development and Quality
% Assurance", E. Sexton-Kennedy
% https://cms-mgt-conferences.web.cern.ch/cms-mgt-conferences/conferences/pres_display.aspx?cid=462&pid=2344

\item Evolution with architectures (32bit/64bit, compilers)
% Various ACAT/CHEP presentations from P.Elmer, M.Kortelainen

\item CPU and I/O optimization?
% Various ACAT/CHEP presentations from P.Elmer, M.Kortelainen

\item Documentation?
% Surely we have documentation?
% CHEP12 "Developing CMS software documentation system" M. Stankevicius
% https://cms-mgt-conferences.web.cern.ch/cms-mgt-conferences/conferences/pres_display.aspx?cid=665&pid=4506

\end{itemize}


\section{Computing Implementation} 

\subsection{Distributed computing infrastructure}

% CHEP12 "Trying to Predict the Future - Resource Planning and Allocation
% in CMS" P. Kreuzer
% https://cms-mgt-conferences.web.cern.ch/cms-mgt-conferences/conferences/pres_display.aspx?cid=665&pid=4582

% CHEP10 "Experience with the CMS Computing Model from commissioning to
% collisions" D. Bonacorsi
% https://cms-mgt-conferences.web.cern.ch/cms-mgt-conferences/conferences/pres_display.aspx?cid=462&pid=2306

% CHEP10 "Monitoring the Readiness and Utilization of the Distributed CMS
% Computing Facilities during the first year LHC running" J. Hernandez
% https://cms-mgt-conferences.web.cern.ch/cms-mgt-conferences/conferences/pres_display.aspx?cid=462&pid=2332
% Note that we don't currently have monitoring in here as an explicit
% topic.  But it probably needs to be discussed in operations, at the very least.

% CHEP10 "CMS Distributed Computing Integration in the LHC sustained
% operations era" C. Grandi
% https://cms-mgt-conferences.web.cern.ch/cms-mgt-conferences/conferences/pres_display.aspx?cid=462&pid=2264

From the very start, CMS planned on having a distributed computing
infrastructure.  While HEP collaborations have historically relied on a single
large computational resource located at the host laboratory for virtually all of
their data processing and storage needs, a facility on this scale would
have been expensive to operate, especially when power and cooling costs are
considered.  Instead, computing for the LHC experiments has taken place at
sites distributed around the world.  The distributed sites have given
greater prominence to the LHC experiments within their participating
nations, which has led to greater funding and more engagement from local
computing experts, who would not have had a chance to participate had all
the computing been done at CERN.

The CMS distributed computing infrastructure is a tiered set of computing
sites connected through networks that follows the MONARC
model~\cite{MONARC}. The computing infrastructure is organized in a
hierarchy starting with CERN at its origin called Tier 0 (T0).  On the next
level, seven regional computing centers called Tier-1 (T1) sites form the
backbone of the system, followed by Tier-2 (T2) and Tier-3 (T3) sites at
universities and/or research institutes.  

Sites at each tier in the hierarchy are configured to support the workflows
that take place at that tier.  The T0 site is responsible for data
recording and first-pass (``prompt'') data reconstruction, and also for
performing calibrations that are needed for the reconstruction.  Prompt
reconstruction starts 48 hours after events have been recorded to
incorporate updated calibrations.  The production of calibration constants
is described in Section~\ref{sec:constants}.  All events are stored in the
RAW format on tape at the T0 site.  \fixme{We'll need to be sure that RAW
  etc. are defined earlier on; this sounds like software.}  The T0 facility
thus needs substantial processing and storage resources.  At the end of Run
1, the CMS T0 facility had \fixme{Get all the T0 parameters!}.

The primary workflows at T1 sites are the simulation of events and the
re-reconstruction of both detector and simulation events.  Simulation is a
processing-intensive workflow, requires little or no input data, but has
large output data.  NOW HERE!!

All sites are interconnected via
dedicated or general purpose scientific networks. The original design
separated the T2 sites into groups that were associated and connected
to only one T1. During LHC Run 1, the strength and reliability of the
networks allowed for a more flexible setup, realizing a full-mesh network
topology where every T2 site is connected to every T1 site and
every other T2 site, as shown in Figure~\ref{fig:distributed_topology}.

\begin{figure}
\begin{center}
\includegraphics[width=.5\textwidth]{figs/distributed_topology}
\end{center}
\caption{The tiered CMS computing infrastructure showing the Tier-0, Tier-1 and Tier-2 levels interconnected with dedicated or general purpose scientific networks in a full-mesh network topology.
  \label{fig:distributed_topology}}
\end{figure}

T1 sites provide access to large CPU farms and a mass storage system
(MSS) with disk caches and tape libraries. They are different from T2
sitess in two main aspects. T1 sites provide continuous (``24/7'')
support operation, while T2 sites are operated during business hours
and unattended in between. In addition, T2 sites do not have tape
libraries at the sites and operate purely with disk caches.  T3 sites
are owned and managed by individual institutes, at a support level of their
choosing.

The network between the sites is the backbone of the CMS computing
infrastructure, as its operation relies on transferring files among the
sites for access. The network between the T0 and T1 sites is the dedicated
LHC Optical Private Network (LHCOPN).~\fixme{Does this get a
  reference?}. The T2 sites are connected through general purpose
scientific networks in their host countries. The main data flows can be
separated into archiving and serving. Archiving is tape based where related
files are grouped by physics content on separate sets of tape cartridges
(known as tape families) to optimize writing, recall, and
recycling. Serving is disk based. The main data flows are as follows:

\subsection{Core computing services}
\label{sec:corecomponents}

\subsubsection{Constants creation and distribution}
\label{sec:constants}
% PCL, Frontier

% CHEP10 "Alignment & calibration experience under LHC data-taking
% conditions in the CMS experiment" R. Mankel
% https://cms-mgt-conferences.web.cern.ch/cms-mgt-conferences/conferences/pres_display.aspx?cid=462&pid=2324

% CHEP12 "Handling of time-critical Conditions Data in the CMS experiment -
% Experience of the first year of data taking" G. Govi
% https://cms-mgt-conferences.web.cern.ch/cms-mgt-conferences/conferences/pres_display.aspx?cid=665&pid=4571

% CHEP12  "Comparison of the Frontier Distributed Database Caching System with NoSQL Databases" D. Dykstra
% https://cms-mgt-conferences.web.cern.ch/cms-mgt-conferences/conferences/pres_display.aspx?cid=665&pid=4418

% CHEP12 "CMS experience with online and offline Databases" A. Pfeiffer
% https://cms-mgt-conferences.web.cern.ch/cms-mgt-conferences/conferences/pres_display.aspx?cid=665&pid=4444

% CHEP12 "Operational Experience with the Frontier System in CMS"
% https://cms-mgt-conferences.web.cern.ch/cms-mgt-conferences/conferences/pres_display.aspx?cid=665&pid=4540

% CHEP10 "Time-critical database condition data handling in the CMS
% experiment during the first data taking period" S. Di Guida
% https://cms-mgt-conferences.web.cern.ch/cms-mgt-conferences/conferences/pres_display.aspx?cid=462&pid=2300

% CHEP10 "CMS Online Database experience with first data" M. Janulis
% https://cms-mgt-conferences.web.cern.ch/cms-mgt-conferences/conferences/pres_display.aspx?cid=462&pid=2310


\subsubsection{Data transfers}
% PhEDEx

% CHEP13 "The CMS Data Management System" N. Magini
% https://cms-mgt-conferences.web.cern.ch/cms-mgt-conferences/conferences/pres_display.aspx?cid=1085&pid=7057

% CHEP12 "CMS Data Transfer operations after the first years of LHC
% collisions" R. Kaselis
% https://cms-mgt-conferences.web.cern.ch/cms-mgt-conferences/conferences/pres_display.aspx?cid=665&pid=4537

% CHEP12 "Performance studies and improvements of CMS Distributed Data
% Transfers", J. Flix
% https://cms-mgt-conferences.web.cern.ch/cms-mgt-conferences/conferences/pres_display.aspx?cid=665&pid=4536

% CHEP10 "Large Scale Commissioning and Operational Experience with Tier-2
% to Tier-2 Data Transfer Links in CMS" J. Letts
% https://cms-mgt-conferences.web.cern.ch/cms-mgt-conferences/conferences/pres_display.aspx?cid=462&pid=2265

% CHEP10 "Improving CMS data transfers among its distributed Computing
% Facilities" N. Magini
% https://cms-mgt-conferences.web.cern.ch/cms-mgt-conferences/conferences/pres_display.aspx?cid=462&pid=2323

% CHEP12 "From toolkit to framework - the past and future evolution of
% PhEDEx" T. Wildish
% https://cms-mgt-conferences.web.cern.ch/cms-mgt-conferences/conferences/pres_display.aspx?cid=665&pid=4552


\subsubsection{Data management}
% DBS
% CHEP10 "Data Aggregation System, an information retrieval on demand over
% relational and non-relational distributed data sources." V. Kuznetsov
% https://cms-mgt-conferences.web.cern.ch/cms-mgt-conferences/conferences/pres_display.aspx?cid=462&pid=2317


\subsubsection{Software distribution}
% CVMFS (and predecessors)

\subsubsection{Submission tools}
% glite, HTCondor_g, glide-in


\subsection{Computing operations and monitoring}
\label{sec:compops}

\subsubsection{Operations model}

\subsubsection{Site monitoring}

\subsubsection{Job monitoring}


\subsection{Workflow execution}
\label{sec:workflow}

\subsubsection{Production workflow management}
% WMAgent
% CHEP12 "The CMS workload management system" S. Wakefield
% https://cms-mgt-conferences.web.cern.ch/cms-mgt-conferences/conferences/pres_display.aspx?cid=665&pid=4454

% CHEP12 "A new era for central processing and production in CMS"
% E. Fajardo Hernandez
% https://cms-mgt-conferences.web.cern.ch/cms-mgt-conferences/conferences/pres_display.aspx?cid=665&pid=4402

% CHEP12 "CMS resource utilization and limitations on the grid after the
% first two years of LHC collisions" K. Bloom
% https://cms-mgt-conferences.web.cern.ch/cms-mgt-conferences/conferences/pres_display.aspx?cid=665&pid=4542

% CHEP10 "Measuring and Understanding Computing Resource Utilization in
% CMS" J. Letts
% https://cms-mgt-conferences.web.cern.ch/cms-mgt-conferences/conferences/pres_display.aspx?cid=462&pid=2268


\subsubsection{User analysis workflows}
% CRAB

% CHEP12 "CMS Analysis Deconstructed" S. Malik
% https://cms-mgt-conferences.web.cern.ch/cms-mgt-conferences/conferences/pres_display.aspx?cid=665&pid=4399

% CHEP12 "Maintaining and improving of the training program on the analysis
% software in CMS" S. Malik
% https://cms-mgt-conferences.web.cern.ch/cms-mgt-conferences/conferences/pres_display.aspx?cid=665&pid=4570

% CHEP10 "Perspective of User Support for the CMS Collaboration" S. Malik
% https://cms-mgt-conferences.web.cern.ch/cms-mgt-conferences/conferences/pres_display.aspx?cid=462&pid=2263

% CHEP10 "A tour of the CMS Physics Analysis Model" B. Hegner
% https://cms-mgt-conferences.web.cern.ch/cms-mgt-conferences/conferences/pres_display.aspx?cid=462&pid=2309

% CHEP10 "CMS distributed analysis infrastructure and operations:
% experience with the first LHC data" E. Vaandering
% https://cms-mgt-conferences.web.cern.ch/cms-mgt-conferences/conferences/pres_display.aspx?cid=462&pid=2338

% CHEP10 "Design and early experience with promoting user-created data in
% CMS" M. Giffels
% https://cms-mgt-conferences.web.cern.ch/cms-mgt-conferences/conferences/pres_display.aspx?cid=462&pid=2293






\section{Anticipated Evolution for Run 2}
\begin{itemize}
\item Multithreaded framework
%CHEP12 "Study of a Fine Grained Threaded Framework Design" C. Jones
% https://cms-mgt-conferences.web.cern.ch/cms-mgt-conferences/conferences/pres_display.aspx?cid=665&pid=4396

%CHEP12 "Multi-core processing and scheduling performance in CMS"
%J. Hernandez
% https://cms-mgt-conferences.web.cern.ch/cms-mgt-conferences/conferences/pres_display.aspx?cid=665&pid=4365

%CHEP10 "Multicore-aware applications in CMS" C. Jones
% https://cms-mgt-conferences.web.cern.ch/cms-mgt-conferences/conferences/pres_display.aspx?cid=462&pid=2342

\item Software engineering efforts
%CHEP12 "Development and Evaluation of Vectorised and Multi-Core Event
%Reconstruction Algorithms within the CMS Software Framework" D. Piparo
% https://cms-mgt-conferences.web.cern.ch/cms-mgt-conferences/conferences/pres_display.aspx?cid=665&pid=4546

\item Evolution of tiered computing model
% CHEP12 "Evolution of the Distributed Computing Model of the CMS
% experiment at the LHC" C. Grandi
% https://cms-mgt-conferences.web.cern.ch/cms-mgt-conferences/conferences/pres_display.aspx?cid=665&pid=4347

\item Use of data federations, changes in data distribution
%CHEP12 "Implementing data placement strategies for the CMS experiment
%based on a popularity mode" D. Giordano
% https://cms-mgt-conferences.web.cern.ch/cms-mgt-conferences/conferences/pres_display.aspx?cid=665&pid=4443
\item Efforts on opportunistic resources
% CHEP12 "Controlled overflowing of data-intensive jobs from oversubscribed
% sites" I. Sfiligoi
% https://cms-mgt-conferences.web.cern.ch/cms-mgt-conferences/conferences/pres_display.aspx?cid=665&pid=4419
\end{itemize}

\section{Conclusion}
Would be good to get back to the physics -- discuss cases where we got results out quick (Higgs), turned around new samples quickly, got new releases or calibrations out fast.  These are ultimately the measures of our success!  Or put another way, here is where we should clearly emphasize that we’ve met some of the goals described earlier.

